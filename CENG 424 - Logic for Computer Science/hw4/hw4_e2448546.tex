\documentclass[12pt]{article}
\usepackage[utf8]{inputenc}
\usepackage{float}
\usepackage{amsmath}
\usepackage[tableaux]{prooftrees}
\renewcommand*\linenumberstyle[1]{(#1)}

\usepackage[hmargin=3cm,vmargin=6.0cm]{geometry}
\topmargin=-2cm
\addtolength{\textheight}{6.5cm}
\addtolength{\textwidth}{2.0cm}
\setlength{\oddsidemargin}{0.0cm}
\setlength{\evensidemargin}{0.0cm}

\usepackage{indentfirst}
\usepackage{amsfonts}
\usepackage{tikz}
\usepackage{qtree}

\begin{document}

\section*{Student Information}

Name: Kaan Karaçanta \\

ID: 2448546 \\


% 1. If the horse is an animal, then some stableman groom the horse. If the horse is a plant, then no
% stableman groom the horse. Use resolution to show that, if the horse is an animal, then the horse
% isn’t a plant.
\section*{Answer 1}

Let $ horse(x) $ be the predicate that $ x $ is a horse, $animal(x) $ be the predicate that $ x $ is an animal, $ plant(x) $ be the predicate that $ x $ is a plant, $ groom(x, y) $ be the predicate that $ x $ is groomed by some stableman $ y $. \\

Then we can write the given sentences as follows:
\begin{align*}
    & \forall x (horse(x) \wedge animal(x) \Rightarrow \exists y (groom(x, y))) \\
    & \forall x (horse(x) \wedge plant(x) \Rightarrow \neg \exists y (groom(x, y))) \\
\end{align*}

We can write the conclusion and its negation as follows:
\begin{align*}
    & \forall x (horse(x) \wedge animal(x) \Rightarrow \neg plant(x)) \\
    & \exists x (horse(x) \wedge animal(x) \wedge plant(x)) \\
\end{align*}

We can write the resolution as follows:
\begin{align*}
    1. \hspace{1em} & \{ \neg horse(x), \neg animal(x), groom(x, f(x)) \}                                                                   & \text{Premise} \\   
    2. \hspace{1em} & \{ \neg horse(z), \neg plant(z), \neg groom(z, f(z)) \}                                                               & \text{Premise} \\
    3. \hspace{1em} & \{ horse(c) \}                                                                                                        & \text{Negated Goal} \\
    4. \hspace{1em} & \{ animal(c) \}                                                                                                       & \text{Negated Goal} \\
    5. \hspace{1em} & \{ plant(c) \}                                                                                                        & \text{Negated Goal} \\
    6. \hspace{1em} & \{ \neg animal(c), groom(c, f(c)) \}                                                                                  & \text{1, 3} \\
    7. \hspace{1em} & \{ \neg plant(c), \neg groom(c, f(c)) \}                                                                              & \text{2, 3} \\
    8. \hspace{1em} & \{ groom(c, f(c)) \}                                                                                                  & \text{4, 6} \\
    9. \hspace{1em} & \{ \neg groom(c, f(c)) \}                                                                                             & \text{5, 7} \\
    10. \hspace{1em} & \{ \}                                                                                                                & \text{8, 9} \\
\end{align*}

Since we added the negated goal to the set of premises and derived the empty clause, we can say that the conclusion `if the horse is an animal, then the horse is not a plant' is valid.

% 2. Consider the set of premises below to derive the empty clause {} using some resolution strategies.
% T, ¬S ∨ ¬T ∨ ¬R, ¬T ∨ R, S ∨ ¬R
\section*{Answer 2}

% (a) Derive the empty clause {} using unit resolution.
\subsection*{\text{(a)}}

\begin{align*}
    1. \hspace{1em} & \{ T \}                                                                   & \text{T} \\
    2. \hspace{1em} & \{ \neg S, \neg T, \neg R \}                                              & \text{$\neg S \vee \neg T \vee \neg R$} \\
    3. \hspace{1em} & \{ \neg T, R \}                                                           & \text{$\neg T \vee R$} \\
    4. \hspace{1em} & \{ S, \neg R \}                                                           & \text{$S \vee \neg R$} \\
    5. \hspace{1em} & \{ \neg S, \neg R \}                                                      & \text{1, 2} \\
    6. \hspace{1em} & \{ R \}                                                                   & \text{1, 3} \\
    7. \hspace{1em} & \{ S \}                                                                   & \text{4, 6} \\
    8. \hspace{1em} & \{ \neg S \}                                                              & \text{5, 6} \\ 
    9. \hspace{1em} & \{ \}                                                                     & \text{7, 8} \\
\end{align*}

% (b) Derive the empty clause {} using input resolution.
\subsection*{\text{(b)}}

\begin{align*}
    1. \hspace{1em} & \{ T \}                                                                   & \text{T} \\
    2. \hspace{1em} & \{ \neg S, \neg T, \neg R \}                                              & \text{$\neg S \vee \neg T \vee \neg R$} \\
    3. \hspace{1em} & \{ \neg T, R \}                                                           & \text{$\neg T \vee R$} \\
    4. \hspace{1em} & \{ S, \neg R \}                                                           & \text{$S \vee \neg R$} \\
    5. \hspace{1em} & \{ \neg S, \neg R \}                                                      & \text{1, 2} \\
    6. \hspace{1em} & \{ \neg R \}                                                              & \text{4, 5} \\
    7. \hspace{1em} & \{ \neg T \}                                                              & \text{3, 6} \\
    8. \hspace{1em} & \{ \}                                                                     & \text{1, 7} \\
\end{align*}

% (c) Derive the empty clause {} using linear resolution.
\subsection*{\text{(c)}}

\begin{align*}
    1. \hspace{1em} & \{ T \}                                                                   & \text{T} \\
    2. \hspace{1em} & \{ \neg S, \neg T, \neg R \}                                              & \text{$\neg S \vee \neg T \vee \neg R$} \\
    3. \hspace{1em} & \{ \neg T, R \}                                                           & \text{$\neg T \vee R$} \\
    4. \hspace{1em} & \{ S, \neg R \}                                                           & \text{$S \vee \neg R$} \\
    5. \hspace{1em} & \{ \neg S, \neg R \}                                                      & \text{1, 2} \\
    6. \hspace{1em} & \{ \neg R \}                                                              & \text{4, 5} \\
    7. \hspace{1em} & \{ \neg T \}                                                              & \text{3, 6} \\
    8. \hspace{1em} & \{ \}                                                                     & \text{1, 7} \\
\end{align*}

% 3. Derive the empty clause {} using ordered resolution from the set of premises below.
% R ∨ P ∨ ¬Q, ¬P ∨ R, ¬Q ∨ ¬R, Q
\section*{Answer 3}

\begin{align*}
    1. \hspace{1em} & \langle \neg Q, P, R \rangle                & \text{Premise} \\
    2. \hspace{1em} & \langle \neg P, R \rangle                   & \text{Premise} \\
    3. \hspace{1em} & \langle \neg R, \neg Q \rangle              & \text{Premise} \\
    4. \hspace{1em} & \langle Q \rangle                           & \text{Premise} \\
    5. \hspace{1em} & \langle P, R \rangle                        & \text{1, 4} \\
    6. \hspace{1em} & \langle R \rangle                           & \text{2, 5} \\
    7. \hspace{1em} & \langle \neg Q \rangle                      & \text{3, 6} \\  
    8. \hspace{1em} & \langle \rangle                             & \text{4, 7} \\  
\end{align*}


\end{document}
