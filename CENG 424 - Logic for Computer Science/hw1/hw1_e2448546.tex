\documentclass[12pt]{article}
\usepackage[utf8]{inputenc}
\usepackage{float}
\usepackage{amsmath}
\usepackage[tableaux]{prooftrees}
\renewcommand*\linenumberstyle[1]{(#1)}

\usepackage[hmargin=3cm,vmargin=6.0cm]{geometry}
\topmargin=-2cm
\addtolength{\textheight}{6.5cm}
\addtolength{\textwidth}{2.0cm}
\setlength{\oddsidemargin}{0.0cm}
\setlength{\evensidemargin}{0.0cm}
\usepackage{indentfirst}
\usepackage{amsfonts}
\usepackage{tikz}
\usepackage{qtree}

\begin{document}

\section*{Student Information}

Name: Kaan Karaçanta \\

ID: 2448546 \\


\section*{Answer 1}

\paragraph{a)} $ A \rightarrow B $ and $ \neg (A \wedge \neg B) $
% give the truth table to show the equivalence here show A, B, not B, A -> B, A ^ not B, not (A ^ not B)

\begin{table}[H]
    \centering
    \begin{tabular}{|l|l|l|l|l|l|}
        \hline
        A & B & $\neg B$ & $ A \rightarrow B $ & $ (A \wedge \neg B) $ & $ \neg (A \wedge \neg B) $ \\ \hline
        T & T & F        & T                   & F                     & T                          \\ \hline
        T & F & T        & F                   & T                     & F                          \\ \hline
        F & T & F        & T                   & F                     & T                          \\ \hline
        F & F & T        & T                   & F                     & T                          \\ \hline
    \end{tabular}

    \caption{Truth Table for $ A \rightarrow B $ and $ \neg (A \wedge \neg B) $}
    \label{t1}
\end{table}

The truth table shows that $ A \rightarrow B $ and $ \neg (A \wedge \neg B) $ are equivalent.

\paragraph{b)} $ A \leftrightarrow B $ and $ (\neg A \vee B) \wedge (\neg B \vee A) $

\begin{table}[H]
    \centering
    \begin{tabular}{|l|l|l|l|l|l|l|l|}
        \hline
        A & B & $\neg A$ & $\neg B$ & $ A \leftrightarrow B $ & $ (\neg A \vee B) $ & $ (\neg B \vee A) $ & $ (\neg A \vee B) \wedge (\neg B \vee A) $ \\ \hline
        T & T & F        & F        & T                       & T                   & T                   & T                                          \\ \hline
        T & F & F        & T        & F                       & F                   & T                   & F                                          \\ \hline
        F & T & T        & F        & F                       & T                   & F                   & F                                          \\ \hline
        F & F & T        & T        & T                       & T                   & T                   & T                                          \\ \hline
    \end{tabular}

    \caption{Truth Table for $ A \leftrightarrow B $ and $ (\neg A \vee B) \wedge (\neg B \vee A) $}
    \label{t2}
\end{table}

The truth table shows that $ A \leftrightarrow B $ and $ (\neg A \vee B) \wedge (\neg B \vee A) $ are equivalent.

\paragraph{c)} $ A \rightarrow (\neg A \rightarrow B) $ and $ 1 $
% give the truth table to show the equivalence here show A, not A, B, not A -> B, A -> (not A -> B)

\begin{table}[H]
    \centering
    \begin{tabular}{|l|l|l|l|l|l|}
        \hline
        A & $\neg A$ & B & $ \neg A \rightarrow B $ & $ A \rightarrow (\neg A \rightarrow B) $ & $ 1 $ \\ \hline
        T & F        & T & T                        & T                                        & T     \\ \hline
        T & F        & F & T                        & T                                        & T     \\ \hline
        F & T        & T & T                        & T                                        & T     \\ \hline
        F & T        & F & F                        & T                                        & T     \\ \hline
    \end{tabular}

    \caption{Truth Table for $ A \rightarrow (\neg A \rightarrow B) $ and $ 1 $}
    \label{t3}
\end{table}

The truth table shows that $ A \rightarrow (\neg A \rightarrow B) $ and $ 1 $ are equivalent.

\newpage

\paragraph{d)} $ (A \vee \neg B) \rightarrow C $ and $ (\neg A \wedge B) \vee C $
% give the truth table to show the equivalence here show A, B, C, not A, not B, A v not B, not A ^ B, A v not B -> C, not A ^ B v C

\begin{table}[H]
    \centering
    \begin{tabular}{|l|l|l|l|l|l|l|l|l|}
        \hline
        A & B & C & $\neg A$ & $\neg B$ & $ A \vee \neg B $ & $ \neg A \wedge B $ & $ (A \vee \neg B) \rightarrow C $ & $ (\neg A \wedge B) \vee C $ \\ \hline
        T & T & T & F        & F        & T                 & F                   & T                                 & T                          \\ \hline
        T & T & F & F        & F        & T                 & F                   & F                                 & F                          \\ \hline
        T & F & T & F        & T        & T                 & F                   & T                                 & T                          \\ \hline
        T & F & F & F        & T        & T                 & F                   & F                                 & F                          \\ \hline
        F & T & T & T        & F        & F                 & T                   & T                                 & T                          \\ \hline
        F & T & F & T        & F        & F                 & T                   & T                                 & T                          \\ \hline
        F & F & T & T        & T        & T                 & F                   & T                                 & T                          \\ \hline
        F & F & F & T        & T        & T                 & F                   & F                                 & F                          \\ \hline
    \end{tabular}

    \caption{Truth Table for $ (A \vee \neg B) \rightarrow C $ and $ (\neg A \wedge B) \vee C $}
    \label{t4}
\end{table}

The truth table shows that $ (A \vee \neg B) \rightarrow C $ and $ (\neg A \wedge B) \vee C $ are equivalent.

% convert logical forms to conjunctive normal form (CNF) in this question
\section*{Answer 2}

% A ∧ (¬A → A)
\paragraph{a)} $ A \wedge (\neg A \rightarrow A) $

\begin{align*}
    A \wedge (\neg A \rightarrow A) & \equiv A \wedge (\neg \neg A \vee A) & \text{(a)} \\
                                    & \equiv A \wedge (A \vee A)           & \text{(b)} \\
                                    & \equiv A \wedge A                    & \text{(c)} \\
                                    & \equiv A                             & \text{(d)} \\
\end{align*}

% (A → B) → ((A → ¬B) → ¬A)
\paragraph{b)} $ (A \rightarrow B) \rightarrow ((A \rightarrow \neg B) \rightarrow \neg A) $

\begin{align*}
    (A \rightarrow B) \rightarrow ((A \rightarrow \neg B) \rightarrow \neg A)  & \equiv (\neg A \vee B) \rightarrow ((\neg A \vee \neg B) \rightarrow \neg A) & \text{(a)} \\
                                                                               & \equiv (\neg A \vee B) \rightarrow (\neg (\neg A \vee \neg B) \vee \neg A)   & \text{(b)} \\
                                                                               & \equiv (\neg A \vee B) \rightarrow ((A \wedge B) \vee \neg A)                & \text{(c)} \\
                                                                               & \equiv (\neg A \vee B) \rightarrow ((A \vee \neg A) \wedge (B \vee \neg A))  & \text{(d)} \\
                                                                               & \equiv (\neg A \vee B) \rightarrow (T \wedge (B \vee \neg A))                & \text{(e)} \\
                                                                               & \equiv (\neg A \vee B) \rightarrow (B \vee \neg A)                           & \text{(f)} \\
                                                                               & \equiv (\neg A \vee B) \rightarrow (\neg A \vee B)                           & \text{(g)} \\
                                                                               & \equiv T                                                                     & \text{(h)} \\
\end{align*}

% (A → (B ∨ ¬C)) ∧ ¬A ∧ B
\paragraph{c)} $ (A \rightarrow (B \vee \neg C)) \wedge \neg A \wedge B $

\begin{align*}
    (A \rightarrow (B \vee \neg C)) \wedge \neg A \wedge B & \equiv (\neg A \vee (B \vee \neg C)) \wedge \neg A \wedge B  & \text{(a)} \\
                                                           & \equiv (\neg A \vee B \vee \neg C) \wedge \neg A \wedge B    & \text{(b)} \\
\end{align*}

% Construct a semantic tableaux (use only rules defined in section 4) to show that the following logical forms mutually consistent or not.
% Semantic tableaux for this question:
% 1. A ∧ B: it can be extended to form a new tableau by adding both A and B below to the branch
% containing A ∧ B.
% 2. A ∨ B: it can be extended to form a new tableau by adding two new branches, one containing A
% and the other containing B.
% 3. A → B: it can be extended to form a new tableau by adding two new branches, one containing ¬A
% and the other containing B.
% 4. A ↔ B: it can be extended to form a new tableau by adding two new branches, one containing
% A ∧ B and the other containing ¬A ∧ ¬B.
% 5. ¬¬A: it can be extended to form a new tableau by adding A below to the branch containing ¬¬A.
% 6. ¬(A ∧ B): it can be extended to form a new tableau by adding two new branches, one containing
% ¬A and the other containing ¬B.
% 7. ¬(A∨B): it can be extended to form a new tableau by adding both ¬A and ¬B below to the branch
% containing ¬(A ∨ B).
% 8. ¬(A → B): it can be extended to form a new tableau by adding both A and ¬B below to the branch
% containing A ∧ B.
% 9. ¬(A ↔ B): it can be extended to form a new tableau by adding two new branches, one containing
% A ∧ ¬B and the other containing ¬A ∧ B.
% 10. Finally, and most importantly, whenever a logical form A and its negation ¬A appear in a branch
% of a tableau, an inconsistency is indicated in that branch and it is said to be ’closed’, i.e. it is not
% further extended. This is because A and ¬A cannot both be true at the same time.

\section*{Answer 3}
% the logical forms to be considered mutual consistance: ¬A ∧ B, ¬(B ∧ C), C ∨ D, ¬(¬A → D)  (use semantic tableaux)

% $ \neg A \wedge B, \neg (B \wedge C), C \vee D, \neg (\neg A \rightarrow D) $ \\
% write the semantic tableaux like below with writing (a), (b), (c), (d) for each line on the right side

\begin{tableau}
    {
        line no sep= 1.5cm,
        just sep= 1.5cm, % Set separation of justification
        for tree={s sep'=10mm},
        close with=\text{closed}
    }
    [\neg A \land B
        [\neg (B \land C)
            [C \lor D
                [\neg (\neg A \rightarrow D)
                    [\neg A, just={From (1)}
                        [B, just={From (1)}
                            [\neg A, just={From (4)}
                                [\neg D, just={From (4)}
                                    [C, just={Alternatives from (3)}
                                        [\neg B, close, just={Alternatives from (2)}]
                                        [\neg C, close]
                                    ]
                                    [D, close]
                                ]
                            ]
                        ]
                    ]
                ]
            ]
        ]
    ]
]
\end{tableau}

This semantic tableaux shows that $ \neg A \wedge B, \neg (B \wedge C), C \vee D, \neg (\neg A \rightarrow D) $ are not mutually consistent.


\end{document}
